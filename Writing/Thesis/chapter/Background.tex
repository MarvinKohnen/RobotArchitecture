\chapter{Related Work / Background}
A thesis usually introduces relevant scientific work in two areas: On the one hand, your topic should be related to current scientific work in that area or addressing a similar research question (research that is addressing a similar research question and answering a similar motivation --- the `why' question). On the other hand, a background part introduces relevant background on a more methodical level required for the specific approach (addressing the `what' and `how' questions) and laying a theoretical foundation for the own approach.

\section{Related Work (Literature Review)}
One core part of your Master thesis \footnote{In a Bachelor thesis this is often briefly addressed in the introduction and mostly to point out how the approach is positioned in the field as such.} is a review of work related to your topic: What is the current consensus on the chosen topic and what have other researcher done in that area. This part sets up a foundation in which you demonstrate that you are capable of working scientifically, grounding your own research in the general field. In general, this section should start and provide a general overview and categorization of the theoretical/ conceptual foundations of the prior research. As a starting point, it introduces general distinctions and perspectives of the research area, briefly describing each branch and differentiating these. 

Ideally, you will identify a gap in current research that your work intends to fill. This speaks to the importance of your work. You might therefore refine your research goal in this section, for example you will explain how you compare and measure your approach in order to demonstrate its effectiveness. 

Overall, when you are using a dedicated section for literature review, start with a brief paragraph that describes the organization of that chapter and finish with a summarizing section that relates this to your own study.

\subsection*{Tips for Related Work}
\begin{itemize}
    \item Always use a systematic approach to search for literature, including databases and reference lists of relevant studies.
    \item Show both sides of the coin and make sure to cover the classics in your field.
    \item Organize your literature review thematically or methodologically to provide a coherent narrative.
    \item Be concise but thorough in summarizing the literature, and make sure to highlight how it relates to your research question.
    \item Present everything in a clear and structured manner, ensuring the source credibility is evaluated.
\end{itemize}
% Raus?

\section{Background}
The background section has the goal to equip the reader with a basic understanding as required to understand the method section. While a background section often recaps and introduces some basic methodological approaches, it is not aimed to provide a full account. Instead, it should, on the one hand, provide a basic understanding for the intended reader and in particular should emphasize aspects that will be relevant for the remainder of the thesis. On the other hand, it should refer to further literature.

%The background work should help the reader understand the main approaches inside the field as well as point out where there are gaps in the literature. 

In a second step, it should provide more detail only in those areas that directly influence the thesis and are required for further understanding, e.g., when methods are later-on used or integrated into the own approach. During writing, one has to be selective on how much detail should be provided and in many cases refer to the original literature (e.g., don't provide a detailed theoretical introduction of computation of neural networks when your focus is on applying these).

\subsection*{Tips and Guiding Questions for Background Research}
\begin{itemize}
    \item What are the key terms, concepts, and problems addressed by the authors?
    \item Position the source already in relation to your specific topic?
\end{itemize}

\chapter{Introduction}
\label{ch:intro}

The introduction should give an overview of the main points and basics. It should answer the questions of 
why the topic is interesting, what is being studied, and how it will be addressed.

\section{Motivation}

The introduction should provide a motivation for the thesis answering the overarching question as to why this is interesting and relevant.

\subsection*{Guiding Questions}
\begin{itemize}
    \item Why does this topic matter? How can your research potentially benefit the field or society?
    \item What gap in the current literature does your study aim to fill?
    \item What personal interests or experiences led you to choose this topic?
\end{itemize}

\section{Research Questions}
Secondly, the introduction should specify the research question as it guides the study's direction and scope. It should be clear, focused, and answerable within the constraints of the resources. This should explain what you are studying. 

\subsection*{Guiding Questions for Research Question Development}
\begin{itemize}
    \item What is the specific problem or issue your research aims to address?
    \item Is your research question narrow enough to be thoroughly explored within the scope of your thesis?
    \item What are the key variables or concepts included in your research question?
    \item How does your research question align with the existing body of knowledge?
\end{itemize}

In many cases, there is a more general research question and in addition there are multiple questions or hypotheses derived from this.

\section{Overview}
Last, the introduction should provide a brief overview on each chapter (single sentence).

\section*{Tips}
\begin{itemize}
    \item To answer the `how' question, you could briefly explain how you plan to reach your research goal. While this will be detailed mainly in the methods chapter, it can be helpful to give a quick overview here.
    \item In general, this might not be the first part you write. Your expose might be a good starting point for the introduction and you should aim for an early draft for the research questions. But otherwise, the introduction is something you should rewrite at the end --- it should point the reader into the correct direction for all your work and therefore benefits if you adjust the focus after your work is done. Also, check that the introduction and conclusion fit well together.
\end{itemize}

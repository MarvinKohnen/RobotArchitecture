\chapter{Discussion}
\label{ch:discussion}

The Discussion chapter is critical for demonstrating your understanding of the research context and your ability to critically analyze your own findings. You should explain the meaning of your results, considering their limitations, and discussing how they align or contrast with previous studies. Further, you should relate your findings to your hypotheses and expectations which might include discussing reasons for why the results turned out differently or inconclusive. Consider the implications of your findings for the theoretical framework within which your research operates. How do they advance, challenge, or refine existing knowledge?

\section{Tips for an Effective Discussion}
\begin{itemize}
    \item \textbf{Be Balanced:} Present both strengths and limitations of your study to provide a balanced view.
    \item \textbf{Stay Focused:} Keep the discussion relevant to your research questions and objectives.
    \item \textbf{Use Evidence:} Support your arguments with references from the literature, ensuring that your conclusions are grounded in evidence and relate to your findings.
    \item \textbf{Clarify Significance:} Clearly articulate the significance of your findings in the context of the field.
    \item \textbf{Avoid Speculation:} Stick to what the data supports. Speculative comments should be clearly identified as such and founded on logical reasoning.
\end{itemize}
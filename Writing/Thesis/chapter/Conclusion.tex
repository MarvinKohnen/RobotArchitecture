\chapter{Conclusion}
\label{ch:conclusion}

The conclusion chapter should summarize the key findings of your study and draw out implications and contributions of your work. It should connect the different elements of your research, reinforcing the importance of the results and their impact on the field in general. In particular, the conclusion should reconnect with the introduction and the original research question and should evaluate if the goals of the work have been reached.

If applicable, outline practical implications of the findings: How can this research be applied in real-world scenarios? But also, acknowledge any limitations of the study and suggest how this might be addressed in future work. Subsequent research questions should be clearly identified and follow-up studies pointed out.

Last, the thesis should end by summarizing key insights and stressing the importance of these.

\section{Tips for the Conclusion}
\begin{itemize}
    \item Recap the main results and address the research questions. This should connect the findings back to the objectives stated in the introduction.
    \item Highlight the contributions of the research --- acknowledge the limitations of the study.
    \item Discuss the practical implications of the findings --- suggest directions for future research.
    \item Avoid introducing new information.
    \item Keep the conclusion concise and focused.
    \item Discussion and Conclusion are often combined into one section (when these are not too long).
\end{itemize}
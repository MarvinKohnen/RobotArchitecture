%---------------------------------------------------------------------------
% Vorlage für Bachelorarbeiten, Masterarbeiten
% der AG Autonome Intelligente Systeme
% Basierend auf der Vorlage der AG PRIA an der Universität Münster
% (Autoren dort: Daniel Tenbrinck, Fabian Gigengack, Michael Schmeing, Lucas Franek, Andreas Nienkötter
% Dazu Dank an Phil Steinhorst und dessen Vorlage, von der Teile übernommen wurden
%---------------------------------------------------------------------------

\documentclass[%
  paper=A4,               % Papiergröße
  twoside=true,           % zweiseitiges Layout
  openright,              % Kapitel auf ungerader Seite beginnen
  11pt,                   % Schriftgröße
  bibliography=totoc,     % Literaturverzeichnis im Inhaltsverzeichnis
  %listof=totoc,           % Abb.-/Tab.-verzeichnis im Inhaltsverzeichnis
  titlepage=on,           % Titel auf eigener Seite
  DIV=12,                 % Satzspiegelberechnung
  BCOR=1.5cm,             % Bindungskorrektur
  parskip=half,            % Absatzabstand
  final
]{scrreprt}

% Basics
\usepackage[x11names]{xcolor}

\usepackage{natbib}

% Schriftarten setzen
\usepackage[T1]{fontenc}
%\usepackage[utf8]{inputenc} % UTF-8 Codierung
\usepackage{microtype}
\usepackage{charter}
\usepackage{sourcesanspro}
\usepackage{nimbusmononarrow}
%\renewcommand*\familydefault{\sfdefault}   % aktivieren für serifenlose Schrift

\usepackage{array}
\usepackage{booktabs}\newcommand{\ra}[1]{\renewcommand{\arraystretch}{#1}}
\usepackage{adjustbox}

% Absatzformatierung
\usepackage{setspace}
\onehalfspacing
\setlength{\parindent}{0cm}

% Sprachoptionen
%\usepackage[ngerman]{babel}
\usepackage[english]{babel}
%\usepackage[german=quotes]{csquotes}    % passende Anführungszeichen

% Kopf- und Fußzeile
\usepackage[headsepline=1pt, automark]{scrlayer-scrpage}
\setkomafont{pageheadfoot}{\sffamily}       % Kopfzeile serifenlos
\setkomafont{pagenumber}{\sffamily\Large}   % Seitenzahl serifenlos und etwas größer
\setkomafont{headsepline}{\color{gray}} % adds a gray line under the header
\renewcommand*{\footfont}{\sffamily\color{gray}}

% adds a thick gray line after the chapter number
\renewcommand*{\chapterformat}{%
    \thechapter\enskip
    \textcolor{gray!50}{\rule[-\dp\strutbox]{1.5pt}{\baselineskip}}\enskip
}

% Alles rund um Floats
\usepackage{graphicx}
\usepackage{subfigure} % Um mehrere Bilder in eine figure einzufügen

% Mathe-Pakete
\usepackage{mathtools}
\usepackage{amsthm}
\usepackage{amssymb}
\usepackage{thmtools}

% Definition der Styles für mathematische Definitionen, Sätze, Beweise, etc.
\declaretheoremstyle[                     % Style für Definitionen, Sätze, Behauptungen, etc.
  headfont=\sffamily\bfseries,            % Font für Überschrift
  notefont=\normalfont\sffamily\itshape,  % Font für Bezeichnung in Klammern
  bodyfont=\normalfont,                   % Font für Inhalt
  headformat=\NAME\ \NUMBER\; \NOTE,      % Reihenfolge: Erst Definition/Satz/etc., dann Nummer, dann Bezeichnung
  headpunct={},                           % kein Punkt am Ende der Überschrift
  postheadspace=\newline,                 % Zeilenumbruch nach Überschrift
  mdframed={                              % Gestaltungsoptionen
    skipabove=1em
    skipbelow=1em,
    innerleftmargin=1em,
    innerrightmargin=1em,
    hidealllines=true,
    backgroundcolor=gray!15
  },
  ]{mainstyle}

\declaretheoremstyle[             % Style für Beweise
  headfont=\bfseries\scshape,
  bodyfont=\normalfont,
  headpunct=:,
  postheadspace=2em,
  qed=\qedsymbol
  ]{proofstyle}

% Definition der entsprechenden Umgebungen
\declaretheorem[                  % Umgebung für Definitionen
  name=Definition,                % auszugebender Name
  parent=chapter,                 % Nummerierung mit vorgestellter Kapitelnummer
  style=mainstyle                 % Nutze mainstyle-Definition (siehe oben)
  ]{definition}                   % Name der Umgebung

\declaretheorem[                  % Umgebung für Sätze
  name=Satz,
  sharenumber=definition,         % gemeinsame Nummerierung mit definition
  style=mainstyle
  ]{satz}

\declaretheorem[                  % Umgebung für Beweise
  name=Beweis,
  numbered=no,                    % Beweise sind nicht nummeriert
  style=proofstyle
  ]{beweis}
  
  % -- Definition der einzelnen Umgebungen
\declaretheoremstyle[%
     headfont=\sffamily\bfseries,
     notefont=\normalfont\sffamily,
     bodyfont=\normalfont,
     headformat=\NAME\ \NUMBER\NOTE,
     headpunct=,
     postheadspace=\newline,
     spaceabove=\parsep,spacebelow=\parsep,
     %shaded={bgcolor=gray!20},
     postheadhook=\theorembookmark,
     mdframed={
         backgroundcolor=gray!20,
             linecolor=gray!20,
             innertopmargin=6pt,
             roundcorner=5pt,
             innerbottommargin=6pt,
             skipbelow=\parsep,
             skipbelow=\parsep }
     ]%
{mainstyle}

\declaretheoremstyle[%
     headfont=\sffamily\bfseries,
     notefont=\normalfont\sffamily,
     bodyfont=\normalfont,
     headformat=\NAME\ \NUMBER\NOTE,
     headpunct=,
     postheadspace=\newline,
     spaceabove=15pt,spacebelow=10pt,
     postheadhook=\theorembookmark]%
{mainstyle_unshaded}

\declaretheoremstyle[%
     headfont=\sffamily\bfseries,
     notefont=\normalfont\sffamily,
     bodyfont=\normalfont,
     headformat=\NUMBER\NAME\NOTE,
     headpunct=,
     postheadspace=\newline,
     spaceabove=15pt,spacebelow=10pt,
     % shaded={bgcolor=gray!20},
     postheadhook=\theorembookmark]%
{mainstyle_unnumbered}

\declaretheorem[name=Definition,parent=section,style=mainstyle]{definition_alt}
\declaretheorem[name=Definition,numbered=no,style=mainstyle]{definition*}
\declaretheorem[name=Definition,sharenumber=definition,style=mainstyle_unshaded]{definitionUnshaded}

\declaretheorem[name=Theorem,sharenumber=definition,style=mainstyle]{theorem}
\declaretheorem[name=Theorem,numbered=no,style=mainstyle_unnumbered]{theorem*}

\declaretheorem[name=Proposition,sharenumber=definition,style=mainstyle]{proposition}
\declaretheorem[name=Lemma,sharenumber=definition,style=mainstyle]{lemma}

\declaretheorem[name=Satz,sharenumber=definition,style=mainstyle]{satz_alt}
\declaretheorem[name=Satz,sharenumber=definition,style=mainstyle_unshaded]{satzUnshaded}
\declaretheorem[name=Satz,numbered=no,style=mainstyle_unnumbered]{satz*}

\declaretheorem[name=Korollar,sharenumber=definition,style=mainstyle]{korollar}

\declaretheorem[name=Notation,numbered=no,style=mainstyle_unnumbered]{notation}
\declaretheorem[name=Bemerkung,numbered=no,style=mainstyle_unnumbered]{bemerkung}
\declaretheorem[name=Beispiel,numbered=no,style=mainstyle_unnumbered]{beispiel}
\declaretheorem[name=Beispiele,numbered=no,style=mainstyle_unnumbered]{beispiele} 

% Querverweise
\usepackage{hyperref}
\usepackage{cleveref}

% Quellcode-Listings
\usepackage{listingsutf8}
\usepackage{listing}
\lstset{%
  showspaces=false,
  showstringspaces=true,
  showtabs=false,
  tabsize=2,
  basicstyle=\footnotesize\ttfamily,
  frame=leftline,
  framerule=3pt,
  framexleftmargin=4pt,
  rulecolor=\color{gray},
  numbers=left,
  numberstyle=\color{gray},
  numbersep=15pt,
  commentstyle=\color{Honeydew4},
  keywordstyle=\color{DarkOrchid3},
  stringstyle=\color{Chartreuse4},
  nolol
}

\usepackage{xcolor} % Für Farben
\usepackage{algorithmic} % Für Pseudo-Code
\usepackage{algorithm} % Wrapper für Pseudo-Code
\usepackage[font={small}, labelfont=bf]{caption} % kleine Bildunterschriften
%\usepackage{geometry} % Für Feinanpassungen des Layouts

% Zusätzliches
\usepackage{lipsum}       % Für Platzhalter-Text
\usepackage{todonotes}    % Erinnerungen an noch abzuarbeitende Baustellen

%\usepackage{listings} % Für Code-Listings
%\renewcommand{\lstlistingname}{Quelltext} %Ändert die Überschrift von Listing nach Quelltext

% Einstellungen für Abstand an den Rändern
%\geometry{a4paper,left=35mm,right=35mm,top=20mm,bottom=20mm, includeheadfoot}
